\cleardoublepage
\chapter*{Introduction}

Distributed with every copy of Python, the Standard Library contains
hundreds of modules that provide tools for interacting with the
operating system, interpreter, and Internet – all of them tested and
ready to be used to jump-start the development of your
applications. This book presents selected examples demonstrating how
to use the most commonly used features of the modules that give Python
its ``batteries included'' slogan, taken from the popular Python
Module of the Week (PyMOTW) blog series.

\section*{This Book's Target Audience}

The audience for this book is an intermediate Python programmer, so
although all of the source code is presented with discussion, only a
few cases include line-by-line explanations.  Every section focuses on
the features of the modules, illustrated by the source code and output
from fully independent example programs.  Each feature is presented as
concisely as possible, so the reader can focus on the module or
function being demonstrated without being distracted by the supporting
code.

An experienced programmer familiar with other languages may be able to
learn Python from this book, but it is not intended to be an
introduction to the language.  Some prior experience writing Python
programs will be useful when studying the examples.

Several sections, such as the description of network programming with
sockets or hmac encryption, require domain-specific knowledge.  The
basic information needed to explain the examples is included here, but
the range of topics covered by the modules in the standard library
makes it impossible to cover every topic comprehensively in a single
volume.  The discussion of each module is followed by a list of
suggested sources for more information and further reading, including
online resources, RFC standards documents, and related books.

The current transition from Python 2 to Python 3 is well under way.
All of the source code for the examples has been updated from the
original online versions and tested with Python 3.5, the current
release of the 3.x series at the time of publication. For examples
that work with Python 2, refer to the older edition of the book,
called \textit{The Python Standard Library By Example}.

\section*{How This Book Is Organized}

The book supplements the comprehensive reference guide available on
\sphinxcode{http://docs.python.org}, providing fully functional
example programs to demonstrate the features described there. The
modules are grouped into chapters to make it easy to find an
individual module for reference and browse by subject for more
leisurely exploration. In the unlikely event that you want to read it
through from cover to cover, it is organized to minimize ``forward
references'' to modules not yet covered, although it was not possible
to eliminate them entirely.

\section*{Downloading the Example Code}

The original versions of the articles are available at
\sphinxcode{https://pymotw.com/3/}. Errata for the book and the sample
code are available on the author's web site\\
(\sphinxcode{https://www.doughellmann.com/books/byexample/}).

\section*{The Conventions Used in This Book}

TBD
